\documentclass[a4paper, 12pt]{article}

% PREAMBULO ====================================================================

% Packages ---------------------------------------------------------------------

% Package to Portuguese language
\usepackage[brazil]{babel}
% Package to Figures
\usepackage{graphicx}
\usepackage{tikz}
% Packages to math symbols and expressions
\usepackage{amsfonts, amssymb, amsmath}
% Package to insert code
\usepackage{listings} 
\usepackage{verbatim}
% Package to justify text
\usepackage[document]{ragged2e}
% Package to manage the bibliography
\usepackage[backend=biber, style=numeric, sorting=none]{biblatex}
% Package to facilitate quotations
\usepackage{csquotes}
% Package to use multicols
\usepackage{multicol}
% Para url
\usepackage{url}
% Fira Font
\usepackage[sfdefault, lf]{FiraSans}
% Colors
\usepackage[table]{xcolor}
% Table
\usepackage{tabularray}
% Fill line
%\usepackage{xhfill}

% Configurations ---------------------------------------------------------------

\AtBeginSection{
    \begin{frame}{\secname}
        \tableofcontents[currentsection,hideallsubsections]
    \end{frame}
}

\AtBeginSubsection{
    \begin{frame}{\subsecname}
        \tableofcontents[subsectionstyle=show/shaded/hide, subsubsectionstyle=hide]
    \end{frame}
}

\AtBeginSubsubsection{
    \begin{frame}{\subsubsecname}
        \tableofcontents[subsectionstyle=show/shaded/hide,subsubsectionstyle=show/shaded/hide/hide]
    \end{frame}
}

% Numbering slides
\setbeamertemplate{footline}[frame number]{}

% Getting rid of bottom navigation bars
\setbeamertemplate{navigation symbols}{}

% Margins
\setbeamersize{text margin left=20pt, text margin right=20pt}

% New commands -----------------------------------------------------------------

\NewTblrTableCommand \aula{\SetCell{bg=green!60,fg=white}}
\NewTblrTableCommand \prova{\SetCell{bg=red!80,fg=white}}
\NewTblrTableCommand \feriado{\SetCell{bg=blue!50,fg=white}}

\newcommand{\esta}[1]{\textbf{\underline{#1}}}

\title{Apresentação da Disciplina\\Programação Orientada a Objetos II}
\author{Evandro J.R. Silva}
\date{}

\begin{document}
\maketitle

\section{Dados Gerais}

\begin{table}[h]
    \centering
    \begin{tblr}{vlines, hlines,
        rows = {c}}
        \textbf{Código} & \textbf{Período} & \textbf{Pré-requisito} & \textbf{Créditos} & \textbf{Carga Horária}\\
        CSHNB025 & 4\textordmasculine & Programação Orientada a Objetos I & 1.3.0 & 60\\
    \end{tblr}
\end{table}

\subsection{Ementa}

\begin{itemize}
    \item Interfaces e processamento de eventos.
    \item Programação gráfica.
    \item Manipulação de Arquivos.
    \item Programação concorrente usando Multithreading.
    \item Programação em rede.
    \item Conectividade com bancos de dados.
\end{itemize}

\subsection{Objetivos}

\begin{itemize}
    \item Ensinar a criar interfaces gráficas simples em Python com Tkinter, explorando eventos, layouts e organização de telas.
    \item Manipular arquivos de texto/CSV e aplicar threads em tarefas demoradas, sempre integrando com interfaces Tkinter.
    \item Introduzir noções de rede (cliente/servidor) e persistência com SQLite, integrando ambos a uma interface gráfica.
\end{itemize}

\subsection{Bibliografia}

\subsubsection{PPC}

\begin{itemize}
    \item \textbf{Básica}
        \begin{itemize}
            \item BARNES, D. J., KÖLLING, M. \textbf{Programação Orientada a Objetos com Java: Uma introdução prática usando BLUEJ}. 4 ed. São Paulo: Pearson Prentice Hall, 2009.
            \item DEITEL, H. M., DEITEL, P. J. \textbf{Java: Como programar}. 8 ed. São Paulo: Pearson Prentice Hall, 2010.
            \item FREEMAN, E., FREEMAN, E. \textbf{Use a Cabeça Padrões de Projetos}. 2 ed. Rio de Janeiro: Altabooks, 2007.
        \end{itemize}
    \item \textbf{Complementar}
\end{itemize}

\subsubsection{Livros/Fontes recentes}

\end{document}