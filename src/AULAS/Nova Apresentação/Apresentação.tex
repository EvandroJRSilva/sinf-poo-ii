\documentclass[a4paper, 12pt]{article}

%-------------------------------------------------------------------------------
% PACOTES ----------------------------------------------------------------------

% Idioma ---------------------
\usepackage[brazil]{babel}

% Fonte ----------------------
\usepackage{mlmodern}
%\usepackage[default,regular,black]{sourceserifpro}
%\usepackage{librebaskerville}
%\usepackage[bitstream-charter]{mathdesign}

% Encoding -------------------
\usepackage[T1]{fontenc}
\usepackage[utf8]{inputenc}

% Links ----------------------
\usepackage[colorlinks=true, linkcolor=blue]{hyperref}

% Personalização -------------
% 	Títulos ------------------
\usepackage{titlesec}
%	Layout da página ---------
\usepackage[top=2cm, left=1.5cm, right=1.5cm, bottom=2cm]{geometry}
%	Listagem -----------------
\usepackage{enumitem}
%	Cores --------------------
\usepackage{xcolor}
%	Colunas
\usepackage{multicol}

% Justificação ---------------
\usepackage{microtype}
\usepackage{ragged2e}

% Matemática -----------------
\usepackage{amsmath, amsthm, amssymb, amsfonts}

% Gráficos -------------------
\usepackage{graphicx}

% Tabelas --------------------
\usepackage{tabularray}

% TODO
%	- https://tikz.dev/pgfcalendar

%------------------------------------------------------------------------------
% CONFIGURAÇÕES ---------------------------------------------------------------

%\titleformat{\section}{format}{label}{sep}{before-code}

\title{Apresentação da Disciplina\\Programação Orientada a Objetos II}
\author{Evandro J.R. Silva}
\date{}

\begin{document}
\maketitle

\section{Dados Gerais}

\begin{table}[h]
    \centering
    \begin{tblr}{vlines, hlines,
        rows = {c}}
        \textbf{Código} & \textbf{Período} & \textbf{Pré-requisito} & \textbf{Créditos} & \textbf{Carga Horária}\\
        CSHNB025 & 4\textordmasculine & Programação Orientada a Objetos I & 1.3.0 & 60\\
    \end{tblr}
\end{table}

\subsection{Ementa}

\begin{itemize}
    \item Interfaces e processamento de eventos.
    \item Programação gráfica.
    \item Manipulação de Arquivos.
    \item Programação concorrente usando Multithreading.
    \item Programação em rede.
    \item Conectividade com bancos de dados.
\end{itemize}

\subsection{Objetivos}

\begin{itemize}
    \item Ensinar a criar interfaces gráficas simples em Python com Tkinter, explorando eventos, layouts e organização de telas.
    \item Manipular arquivos de texto/CSV e aplicar threads em tarefas demoradas, sempre integrando com interfaces Tkinter.
    \item Introduzir noções de rede (cliente/servidor) e persistência com SQLite, integrando ambos a uma interface gráfica.
\end{itemize}

\subsection{Bibliografia}

\subsubsection{PPC}

\begin{itemize}
    \item \textbf{Básica}
        \begin{itemize}
            \item BARNES, D. J., KÖLLING, M. \textbf{Programação Orientada a Objetos com Java: Uma introdução prática usando BLUEJ}. 4 ed. São Paulo: Pearson Prentice Hall, 2009.
            \item DEITEL, H. M., DEITEL, P. J. \textbf{Java: Como programar}. 8 ed. São Paulo: Pearson Prentice Hall, 2010.
            \item FREEMAN, E., FREEMAN, E. \textbf{Use a Cabeça Padrões de Projetos}. 2 ed. Rio de Janeiro: Altabooks, 2007.
        \end{itemize}
    \item \textbf{Complementar}
\end{itemize}

\subsubsection{Livros/Fontes recentes}

\end{document}