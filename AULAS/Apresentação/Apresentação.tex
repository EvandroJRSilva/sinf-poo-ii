\documentclass{beamer}

\usetheme{Antibes}

% Loading preambulo
% PREAMBULO ====================================================================

% Packages ---------------------------------------------------------------------

% Package to Portuguese language
\usepackage[brazil]{babel}
% Package to Figures
\usepackage{graphicx}
\usepackage{tikz}
% Packages to math symbols and expressions
\usepackage{amsfonts, amssymb, amsmath}
% Package to insert code
\usepackage{listings} 
\usepackage{verbatim}
% Package to justify text
\usepackage[document]{ragged2e}
% Package to manage the bibliography
\usepackage[backend=biber, style=numeric, sorting=none]{biblatex}
% Package to facilitate quotations
\usepackage{csquotes}
% Package to use multicols
\usepackage{multicol}
% Para url
\usepackage{url}
% Fira Font
\usepackage[sfdefault, lf]{FiraSans}
% Colors
\usepackage[table]{xcolor}
% Table
\usepackage{tabularray}
% Fill line
%\usepackage{xhfill}

% Configurations ---------------------------------------------------------------

\AtBeginSection{
    \begin{frame}{\secname}
        \tableofcontents[currentsection,hideallsubsections]
    \end{frame}
}

\AtBeginSubsection{
    \begin{frame}{\subsecname}
        \tableofcontents[subsectionstyle=show/shaded/hide, subsubsectionstyle=hide]
    \end{frame}
}

\AtBeginSubsubsection{
    \begin{frame}{\subsubsecname}
        \tableofcontents[subsectionstyle=show/shaded/hide,subsubsectionstyle=show/shaded/hide/hide]
    \end{frame}
}

% Numbering slides
\setbeamertemplate{footline}[frame number]{}

% Getting rid of bottom navigation bars
\setbeamertemplate{navigation symbols}{}

% Margins
\setbeamersize{text margin left=20pt, text margin right=20pt}

% New commands -----------------------------------------------------------------

\NewTblrTableCommand \aula{\SetCell{bg=green!60,fg=white}}
\NewTblrTableCommand \prova{\SetCell{bg=red!80,fg=white}}
\NewTblrTableCommand \feriado{\SetCell{bg=blue!50,fg=white}}

\newcommand{\esta}[1]{\textbf{\underline{#1}}}

% Title
\title[AP 00]{Programação Orientada a Objetos II}
% Subtitle
\subtitle{Apresentação da Disciplina}
% Author of the presentation
\author[E.J.R. Silva]{Evandro J.R. Silva}
% date of the presentation
\date{}

\begin{document}

\begin{frame}
    \titlepage
\end{frame}

\begin{frame}{Sumário}
    \tableofcontents
\end{frame}

%===============================================================================
% SECTION 1 ====================================================================
%===============================================================================
\section{Informações gerais}

%-------------------------------------------------------------------------------
% SUBSECTION 1.1 ---------------------------------------------------------------
%-------------------------------------------------------------------------------
\subsection{Ementa \& Objetivos}

\begin{frame}{Ementa}
    \begin{itemize}
        \justifying
        \item Interfaces e processamento de eventos.
        \item Programação gráfica.
        \item Manipulação de Arquivos.
        \item Programação concorrente usando Multithreading.
        \item Programação em rede.
        \item Conectividade com bancos de dados: JDBC.
    \end{itemize}
\end{frame}

\begin{frame}{Objetivos}
    \begin{itemize}
        \justifying
        \item Aprender a criar aplicações gráficas em JavaFX, entendendo o modelo de cena, tratamento de eventos e boas práticas de organização de telas.
        \item Aprender a salvar, carregar e processar dados, e noções de execução paralela, integrando com interfaces gráficas JavaFX.
        \item Introduzir comunicação em rede e persistência com JDBC, integrando os resultados em uma aplicação gráfica em JavaFX.
    \end{itemize}
\end{frame}

%-------------------------------------------------------------------------------
% SUBSECTION 1.2 ---------------------------------------------------------------
%-------------------------------------------------------------------------------
\subsection{Bibliografia}

\begin{frame}{Bibliografia}
    \begin{itemize}
        \item \textbf{Básica - PPC}
        \begin{itemize}
            \justifying
            \scriptsize
            \item BARNES, D. J., KÖLLING, M. \textbf{Programação Orientada a Objetos com Java: Uma introdução prática usando BLUEJ}. 4 ed. São Paulo: Pearson Prentice Hall, 2009.
            \item DEITEL, H. M., DEITEL, P. J. \textbf{Java: Como programar}. 8 ed. São Paulo: Pearson Prentice Hall, 2010.
            \item FREEMAN, E., FREEMAN, E. \textbf{Use a Cabeça Padrões de Projetos}. 2 ed. Rio de Janeiro: Altabooks, 2007.
        \end{itemize}
        \normalsize
        \item \textbf{Bibliografia complementar - PPC}
        \begin{itemize}
            \justifying
            \scriptsize
            \item SIERRA, K.; BATES, B. \textbf{Use a Cabeça! Java}. 1 ed. Rio de Janeiro: AltaBooks, 2005.
            \item HORSTMANN, C. S.; CORNELL, G. \textbf{Core Java 2: Fundamentos}. 7 ed. Rio de Janeiro: Alta Books, 2005.
            \item KURNIAWAN, Budi. \textbf{Java para Web com Servlets, JSP e EJB}. 1 ed. Rio de Janeiro: Ciência Moderna, 2002.
            \item CADENHEAD, Rogers; LEMAY, Laura. \textbf{Aprenda em 21 dias Java 2}. 4 ed. Rio de Janeiro: Elsevier, 2005.
            \item HORSTMANN, C. \textbf{Big Java}. 4 ed. John Wiley e Sons, 2006.
        \end{itemize}
    \end{itemize}
\end{frame}

\begin{frame}{Bibliografia}
    \begin{itemize}
        \item \textbf{Básica - Livros/Fontes mais recentes}
        \begin{itemize}
            \justifying
            \scriptsize
            \item \textbf{Learn Java}. Disponível em \url{https://dev.java/learn/}. Acesso em 19 ago. 2025.
            \item DOWNEY, Allen B.; MAYFIELD, Chris. \textbf{Think Java}. 2. ed., versão 7.1.0. Needham: Green Tea Press, 2020. Disponível em \url{https://greenteapress.com/thinkjava7/thinkjava2.pdf}. Acesso em 19 ago. 2025.
            \item ECK, David J. \textbf{Introduction to Programming Using Java: Version 9.0, Swing Edition}. Disponível em \url{https://math.hws.edu/javanotes-swing/}. Acesso em 19 ago. 2025.
            \item ECK, David J. \textbf{Introduction to Programming Using Java: Version 9.0, JavaFX Edition}. Disponível em \url{https://math.hws.edu/javanotes/}. Acesso em 19 ago. 2025.
        \end{itemize}
        \normalsize
        \item \textbf{Bibliografia complementar - Livros/Fontes mais recentes}
        \begin{itemize}
            \justifying
            \scriptsize
            \item \textbf{Java Notes for Professionals}. Disponível em \url{https://goalkicker.com/JavaBook/JavaNotesForProfessionals.pdf}. Acesso em 19 ago. 2025.
            \item \textbf{Introduction to JDBC}. Disponível em \url{https://www.baeldung.com/java-jdbc}. Acesso em 19 ago. 2025.
            \item \textbf{Accessing data with MySQL}. Disponível em \url{https://spring.io/guides/gs/accessing-data-mysql}. Acesso em 19 ago. 2025.
        \end{itemize}
    \end{itemize}
\end{frame}

%-------------------------------------------------------------------------------
% SUBSECTION 1.3 ---------------------------------------------------------------
%-------------------------------------------------------------------------------
\subsection{Conteúdo Programático}

\begin{frame}{Conteúdo Programático}
    \begin{itemize}
        \item \textbf{Parte I - Interfaces, Eventos e Programação Gráfica}
            \begin{enumerate}
                \justifying
                \item JavaFX
                    \begin{itemize}
                        \item Fundamentos.
                        \item Modelo de eventos e propriedades.
                        \item Layouts.
                        \item Controles básicos.
                        \item Listas e tabelas.
                        \item Validação e feedback.
                        \item Menus, toolbars e atalhos.
                        \item Organização de telas e navegação.
                    \end{itemize}
            \end{enumerate}
    \end{itemize}
\end{frame}

\begin{frame}{Conteúdo Programático}
    \begin{itemize}
        \justifying
        \item \textbf{Parte II - Arquivos e Multithreading}
            \begin{enumerate}
                \justifying
                \item Arquivos de texto simples.
                \item Threads em Java.
                \item Salvando dados de formulários em arquivos.
                \item Importar/exportar dados em CSV.
                \item Leitura de arquivo para análise.
                \item Serialização de objetos.
                \item Threads simulando tarefas demoradas.
                \item Uso de \texttt{Task} e \texttt{Service} no JavaFX.
            \end{enumerate}
    \end{itemize}
\end{frame}

\begin{frame}{Conteúdo Programático}
    \begin{itemize}
        \justifying
        \item \textbf{Parte III - Programação em Rede e Banco de Dados}
            \begin{enumerate}
                \justifying
                \item Noções básicas de rede.
                \item JDBC.
                \item Cliente TCP simples em JavaFX.
                \item Servidor TCP simples.
                \item Conexão SQLite com JDBC em JavaFX.
                \item Inserção de registros no banco via GUI.
                \item Atualização e exclusão de registros via GUI.
                \item Integração GUI + Banco + Rede.
            \end{enumerate}
    \end{itemize}
\end{frame}

%-------------------------------------------------------------------------------
% SUBSECTION 1.3 ---------------------------------------------------------------
%-------------------------------------------------------------------------------
\subsection{Avaliação}

\begin{frame}{Avaliação}
    \begin{itemize}
        \justifying
        \item Ao \textbf{fim de cada unidade}, será realizada uma \textbf{avaliação parcial} dos conteúdos ministrados durante o curso da unidade, \alert{\textbf{totalizando em 03 (três) avaliações}}.
        \item A \textbf{nota de cada avaliação} poderá ser \textbf{composta por um ou mais instrumentos de avaliação}, de acordo com um dos seguintes casos:
        \begin{enumerate}
            \justifying
            \item Uma prova escrita;
            \item Um ou mais trabalhos (individuais ou em grupo);
            \item Um ou mais trabalhos, mais uma prova escrita.
        \end{enumerate}
    \end{itemize}
\end{frame}

\begin{frame}{Avaliação}
    \begin{itemize}
        \justifying
        \item Nos casos em que a \textbf{avaliação} for \textbf{composta por mais de um instrumento}, será realizado o \textbf{somatório} ou a \textbf{média ponderada} das \textbf{notas obtidas em cada instrumento} para compor a \textbf{nota final} de uma \textbf{avaliação parcial}.
        \item Os instrumentos a serem utilizados em cada avaliação serão definidos e informados no decorrer do curso.
    \end{itemize}
\end{frame}

\begin{frame}{Avaliação}
    \begin{itemize}
        \justifying
        \item As \textbf{notas} obedecem a uma escala de \textbf{0,0 (zero)} a \textbf{10,0 (dez)}, contando até a primeira ordem decimal com possíveis arredondamentos.
        \item Considerar-se-á \textbf{aprovado} na disciplina o aluno que obtiver \textbf{assiduidade igual ou superior a 75\%} e a \textbf{média aritmética} nas \underline{avaliações parciais (média parcial)} \textbf{igual ou superior a 7,0 (sete)}
        \begin{itemize}
            \justifying
            \item OU que se submeta a \alert<2>{exame final} e obtenha média aritmética entre a média parcial e exame final (média final) igual ou superior a 6,0 (seis).
            \begin{itemize}
                \justifying
                \item<2> Terá direito de realizar exame final o aluno que satisfaça os requisitos de assiduidade e que obtenha média parcial maior ou igual a 4,0 (quatro) e menor que 7,0 (sete).
            \end{itemize}
        \end{itemize}
    \end{itemize}
\end{frame}

%-------------------------------------------------------------------------------
% SUBSECTION 1.4 ---------------------------------------------------------------
%-------------------------------------------------------------------------------
\subsection{Calendário}

\begin{frame}{Calendário}
    \centering
    \begin{tblr}{c c c}
        \aula AULA & \feriado FERIADO & \prova AVALIAÇÃO
    \end{tblr}
    
    \begin{columns}
        \begin{column}{0.3\textwidth}
            \begin{table}
                \centering
                \textbf{SETEMBRO}\\ \vspace{0.15cm}
                \begin{tblr}{Q[c,m] Q[c,m] Q[c,m] Q[c,m] Q[c,m]}
                    \hline
                    \textbf{S} & \textbf{T} & \textbf{Q} & \textbf{Q} & \textbf{S} \\
                    \hline
                    01 & 02 & 03 & 04 & \aula\esta{05}\\
                    08 & \aula09 & 10 & 11 & \aula12\\
                    15 & \aula16 & 17 & 18 & \aula19\\
                    22 & \aula23 & 24 & 25 & \prova26\\
                    29 & \aula30   &    &    &   \\
                    \hline
                \end{tblr}
            \end{table}
        \end{column}
        
        \begin{column}{0.7\textwidth}
            \begin{itemize}
                \justifying
                \item Apresentação da disciplina.
            \end{itemize}
        \end{column}
    \end{columns}
\end{frame}

\begin{frame}{Calendário}
    \centering
    \begin{tblr}{c c c}
        \aula AULA & \feriado FERIADO & \prova AVALIAÇÃO
    \end{tblr}
    
    \begin{columns}
        \begin{column}{0.3\textwidth}
            \begin{table}
                \centering
                \textbf{SETEMBRO}\\ \vspace{0.15cm}
                \begin{tblr}{Q[c,m] Q[c,m] Q[c,m] Q[c,m] Q[c,m]}
                    \hline
                    \textbf{S} & \textbf{T} & \textbf{Q} & \textbf{Q} & \textbf{S} \\
                    \hline
                    01 & 02 & 03 & 04 & \aula05\\
                    08 & \aula\esta{09} & 10 & 11 & \aula12\\
                    15 & \aula16 & 17 & 18 & \aula19\\
                    22 & \aula23 & 24 & 25 & \prova26\\
                    29 & \aula30   &    &    &   \\
                    \hline
                \end{tblr}
            \end{table}
        \end{column}
        
        \begin{column}{0.7\textwidth}
            \begin{itemize}
                \justifying
                \item Fundamentos do JavaFX.
                \item Modelo de eventos e propriedades.
            \end{itemize}
        \end{column}
    \end{columns}
\end{frame}

\begin{frame}{Calendário}
    \centering
    \begin{tblr}{c c c}
        \aula AULA & \feriado FERIADO & \prova AVALIAÇÃO
    \end{tblr}
    
    \begin{columns}
        \begin{column}{0.3\textwidth}
            \begin{table}
                \centering
                \textbf{SETEMBRO}\\ \vspace{0.15cm}
                \begin{tblr}{Q[c,m] Q[c,m] Q[c,m] Q[c,m] Q[c,m]}
                    \hline
                    \textbf{S} & \textbf{T} & \textbf{Q} & \textbf{Q} & \textbf{S} \\
                    \hline
                    01 & 02 & 03 & 04 & \aula05\\
                    08 & \aula09 & 10 & 11 & \aula\esta{12}\\
                    15 & \aula16 & 17 & 18 & \aula19\\
                    22 & \aula23 & 24 & 25 & \prova26\\
                    29 & \aula30   &    &    &   \\
                    \hline
                \end{tblr}
            \end{table}
        \end{column}
        
        \begin{column}{0.7\textwidth}
            \begin{itemize}
                \justifying
                \item Layouts.
                \item Controles básicos.
            \end{itemize}
        \end{column}
    \end{columns}
\end{frame}

\begin{frame}{Calendário}
    \centering
    \begin{tblr}{c c c}
        \aula AULA & \feriado FERIADO & \prova AVALIAÇÃO
    \end{tblr}
    
    \begin{columns}
        \begin{column}{0.3\textwidth}
            \begin{table}
                \centering
                \textbf{SETEMBRO}\\ \vspace{0.15cm}
                \begin{tblr}{Q[c,m] Q[c,m] Q[c,m] Q[c,m] Q[c,m]}
                    \hline
                    \textbf{S} & \textbf{T} & \textbf{Q} & \textbf{Q} & \textbf{S} \\
                    \hline
                    01 & 02 & 03 & 04 & \aula05\\
                    08 & \aula09 & 10 & 11 & \aula12\\
                    15 & \aula\esta{16} & 17 & 18 & \aula19\\
                    22 & \aula23 & 24 & 25 & \prova26\\
                    29 & \aula30   &    &    &   \\
                    \hline
                \end{tblr}
            \end{table}
        \end{column}
        
        \begin{column}{0.7\textwidth}
            \begin{itemize}
                \justifying
                \item Listas e tabelas.
            \end{itemize}
        \end{column}
    \end{columns}
\end{frame}

\begin{frame}{Calendário}
    \centering
    \begin{tblr}{c c c}
        \aula AULA & \feriado FERIADO & \prova AVALIAÇÃO
    \end{tblr}
    
    \begin{columns}
        \begin{column}{0.3\textwidth}
            \begin{table}
                \centering
                \textbf{SETEMBRO}\\ \vspace{0.15cm}
                \begin{tblr}{Q[c,m] Q[c,m] Q[c,m] Q[c,m] Q[c,m]}
                    \hline
                    \textbf{S} & \textbf{T} & \textbf{Q} & \textbf{Q} & \textbf{S} \\
                    \hline
                    01 & 02 & 03 & 04 & \aula05\\
                    08 & \aula09 & 10 & 11 & \aula12\\
                    15 & \aula16 & 17 & 18 & \aula\esta{19}\\
                    22 & \aula23 & 24 & 25 & \prova26\\
                    29 & \aula30   &    &    &   \\
                    \hline
                \end{tblr}
            \end{table}
        \end{column}
        
        \begin{column}{0.7\textwidth}
            \begin{itemize}
                \justifying
                \item Validação e feedback.
                \item Menus, toolbars e atalhos.
            \end{itemize}
        \end{column}
    \end{columns}
\end{frame}

\begin{frame}{Calendário}
    \centering
    \begin{tblr}{c c c}
        \aula AULA & \feriado FERIADO & \prova AVALIAÇÃO
    \end{tblr}
    
    \begin{columns}
        \begin{column}{0.3\textwidth}
            \begin{table}
                \centering
                \textbf{SETEMBRO}\\ \vspace{0.15cm}
                \begin{tblr}{Q[c,m] Q[c,m] Q[c,m] Q[c,m] Q[c,m]}
                    \hline
                    \textbf{S} & \textbf{T} & \textbf{Q} & \textbf{Q} & \textbf{S} \\
                    \hline
                    01 & 02 & 03 & 04 & \aula05\\
                    08 & \aula09 & 10 & 11 & \aula12\\
                    15 & \aula16 & 17 & 18 & \aula19\\
                    22 & \aula\esta{23} & 24 & 25 & \prova26\\
                    29 & \aula30   &    &    &   \\
                    \hline
                \end{tblr}
            \end{table}
        \end{column}
        
        \begin{column}{0.7\textwidth}
            \begin{itemize}
                \justifying
                \item Organização de telas e navegação.
            \end{itemize}
        \end{column}
    \end{columns}
\end{frame}

\begin{frame}{Calendário}
    \centering
    \begin{tblr}{c c c}
        \aula AULA & \feriado FERIADO & \prova AVALIAÇÃO
    \end{tblr}
    
    \begin{columns}
        \begin{column}{0.3\textwidth}
            \begin{table}
                \centering
                \textbf{SETEMBRO}\\ \vspace{0.15cm}
                \begin{tblr}{Q[c,m] Q[c,m] Q[c,m] Q[c,m] Q[c,m]}
                    \hline
                    \textbf{S} & \textbf{T} & \textbf{Q} & \textbf{Q} & \textbf{S} \\
                    \hline
                    01 & 02 & 03 & 04 & \aula05\\
                    08 & \aula09 & 10 & 11 & \aula12\\
                    15 & \aula16 & 17 & 18 & \aula19\\
                    22 & \aula23 & 24 & 25 & \prova\esta{26}\\
                    29 & \aula30   &    &    &   \\
                    \hline
                \end{tblr}
            \end{table}
        \end{column}
        
        \begin{column}{0.7\textwidth}
            \Large\centering Primeira Avaliação
        \end{column}
    \end{columns}
\end{frame}

\begin{frame}{Calendário}
    \centering
    \begin{tblr}{c c c}
        \aula AULA & \feriado FERIADO & \prova AVALIAÇÃO
    \end{tblr}
    
    \begin{columns}
        \begin{column}{0.3\textwidth}
            \begin{table}
                \centering
                \textbf{SETEMBRO}\\ \vspace{0.15cm}
                \begin{tblr}{Q[c,m] Q[c,m] Q[c,m] Q[c,m] Q[c,m]}
                    \hline
                    \textbf{S} & \textbf{T} & \textbf{Q} & \textbf{Q} & \textbf{S} \\
                    \hline
                    01 & 02 & 03 & 04 & \aula05\\
                    08 & \aula09 & 10 & 11 & \aula12\\
                    15 & \aula16 & 17 & 18 & \aula19\\
                    22 & \aula23 & 24 & 25 & \prova26\\
                    29 & \aula\esta{30}   &    &    &   \\
                    \hline
                \end{tblr}
            \end{table}
        \end{column}
        
        \begin{column}{0.7\textwidth}
            \begin{itemize}
                \justifying
                \item Arquivos de texto simples.
            \end{itemize}
        \end{column}
    \end{columns}
\end{frame}

\begin{frame}{Calendário}
    \centering
    \begin{tblr}{c c c}
        \aula AULA & \feriado FERIADO & \prova AVALIAÇÃO
    \end{tblr}
    
    \begin{columns}
        \begin{column}{0.3\textwidth}
            \begin{table}
                \centering
                \textbf{OUTUBRO}\\ \vspace{0.15cm}
                \begin{tblr}{Q[c,m] Q[c,m] Q[c,m] Q[c,m] Q[c,m]}
                    \hline
                    \textbf{S} & \textbf{T} & \textbf{Q} & \textbf{Q} & \textbf{S} \\
                    \hline
                    &  & 01 & 02 & \aula\esta{03}\\
                    06 & \aula07 & 08 & 09 & \aula10\\
                    13 & \aula14 & \feriado15 & 16 & \aula17\\
                    20 & \aula21 & 22 & 23 & \aula24\\
                    27 & \aula28 & 29 & 30 & \prova31\\
                    \hline
                \end{tblr}
            \end{table}
        \end{column}
        
        \begin{column}{0.7\textwidth}
            \begin{itemize}
                \justifying
                \item Threads em Java.
            \end{itemize}
        \end{column}
    \end{columns}
\end{frame}

\begin{frame}{Calendário}
    \centering
    \begin{tblr}{c c c}
        \aula AULA & \feriado FERIADO & \prova AVALIAÇÃO
    \end{tblr}
    
    \begin{columns}
        \begin{column}{0.3\textwidth}
            \begin{table}
                \centering
                \textbf{OUTUBRO}\\ \vspace{0.15cm}
                \begin{tblr}{Q[c,m] Q[c,m] Q[c,m] Q[c,m] Q[c,m]}
                    \hline
                    \textbf{S} & \textbf{T} & \textbf{Q} & \textbf{Q} & \textbf{S} \\
                    \hline
                    &  & 01 & 02 & \aula03\\
                    06 & \aula\esta{07} & 08 & 09 & \aula10\\
                    13 & \aula14 & \feriado15 & 16 & \aula17\\
                    20 & \aula21 & 22 & 23 & \aula24\\
                    27 & \aula28 & 29 & 30 & \prova31\\
                    \hline
                \end{tblr}
            \end{table}
        \end{column}
        
        \begin{column}{0.7\textwidth}
            \begin{itemize}
                \justifying
                \item Salvando dados de formulários em arquivos.
            \end{itemize}
        \end{column}
    \end{columns}
\end{frame}

\begin{frame}{Calendário}
    \centering
    \begin{tblr}{c c c}
        \aula AULA & \feriado FERIADO & \prova AVALIAÇÃO
    \end{tblr}
    
    \begin{columns}
        \begin{column}{0.3\textwidth}
            \begin{table}
                \centering
                \textbf{OUTUBRO}\\ \vspace{0.15cm}
                \begin{tblr}{Q[c,m] Q[c,m] Q[c,m] Q[c,m] Q[c,m]}
                    \hline
                    \textbf{S} & \textbf{T} & \textbf{Q} & \textbf{Q} & \textbf{S} \\
                    \hline
                    &  & 01 & 02 & \aula03\\
                    06 & \aula07 & 08 & 09 & \aula\esta{10}\\
                    13 & \aula14 & \feriado15 & 16 & \aula17\\
                    20 & \aula21 & 22 & 23 & \aula24\\
                    27 & \aula28 & 29 & 30 & \prova31\\
                    \hline
                \end{tblr}
            \end{table}
        \end{column}
        
        \begin{column}{0.7\textwidth}
            \begin{itemize}
                \justifying
                \item Importar/exportar dados em CSV.
            \end{itemize}
        \end{column}
    \end{columns}
\end{frame}

\begin{frame}{Calendário}
    \centering
    \begin{tblr}{c c c}
        \aula AULA & \feriado FERIADO & \prova AVALIAÇÃO
    \end{tblr}
    
    \begin{columns}
        \begin{column}{0.3\textwidth}
            \begin{table}
                \centering
                \textbf{OUTUBRO}\\ \vspace{0.15cm}
                \begin{tblr}{Q[c,m] Q[c,m] Q[c,m] Q[c,m] Q[c,m]}
                    \hline
                    \textbf{S} & \textbf{T} & \textbf{Q} & \textbf{Q} & \textbf{S} \\
                    \hline
                    &  & 01 & 02 & \aula03\\
                    06 & \aula07 & 08 & 09 & \aula10\\
                    13 & \aula\esta{14} & \feriado15 & 16 & \aula17\\
                    20 & \aula21 & 22 & 23 & \aula24\\
                    27 & \aula28 & 29 & 30 & \prova31\\
                    \hline
                \end{tblr}
            \end{table}
        \end{column}
        
        \begin{column}{0.7\textwidth}
            \begin{itemize}
                \justifying
                \item Leitura de arquivo para análise.
            \end{itemize}
        \end{column}
    \end{columns}
\end{frame}

\begin{frame}{Calendário}
    \centering
    \begin{tblr}{c c c}
        \aula AULA & \feriado FERIADO & \prova AVALIAÇÃO
    \end{tblr}
    
    \begin{columns}
        \begin{column}{0.3\textwidth}
            \begin{table}
                \centering
                \textbf{OUTUBRO}\\ \vspace{0.15cm}
                \begin{tblr}{Q[c,m] Q[c,m] Q[c,m] Q[c,m] Q[c,m]}
                    \hline
                    \textbf{S} & \textbf{T} & \textbf{Q} & \textbf{Q} & \textbf{S} \\
                    \hline
                    &  & 01 & 02 & \aula03\\
                    06 & \aula07 & 08 & 09 & \aula10\\
                    13 & \aula14 & \feriado15 & 16 & \aula\esta{17}\\
                    20 & \aula21 & 22 & 23 & \aula24\\
                    27 & \aula28 & 29 & 30 & \prova31\\
                    \hline
                \end{tblr}
            \end{table}
        \end{column}
        
        \begin{column}{0.7\textwidth}
            \begin{itemize}
                \justifying
                \item Serialização de objetos.
            \end{itemize}
        \end{column}
    \end{columns}
\end{frame}

\begin{frame}{Calendário}
    \centering
    \begin{tblr}{c c c}
        \aula AULA & \feriado FERIADO & \prova AVALIAÇÃO
    \end{tblr}
    
    \begin{columns}
        \begin{column}{0.3\textwidth}
            \begin{table}
                \centering
                \textbf{OUTUBRO}\\ \vspace{0.15cm}
                \begin{tblr}{Q[c,m] Q[c,m] Q[c,m] Q[c,m] Q[c,m]}
                    \hline
                    \textbf{S} & \textbf{T} & \textbf{Q} & \textbf{Q} & \textbf{S} \\
                    \hline
                    &  & 01 & 02 & \aula03\\
                    06 & \aula07 & 08 & 09 & \aula10\\
                    13 & \aula14 & \feriado15 & 16 & \aula17\\
                    20 & \aula\esta{21} & 22 & 23 & \aula24\\
                    27 & \aula28 & 29 & 30 & \prova31\\
                    \hline
                \end{tblr}
            \end{table}
        \end{column}
        
        \begin{column}{0.7\textwidth}
            \begin{itemize}
                \justifying
                \item Threads simulando tarefas demoradas.
            \end{itemize}
        \end{column}
    \end{columns}
\end{frame}

\begin{frame}{Calendário}
    \centering
    \begin{tblr}{c c c}
        \aula AULA & \feriado FERIADO & \prova AVALIAÇÃO
    \end{tblr}
    
    \begin{columns}
        \begin{column}{0.3\textwidth}
            \begin{table}
                \centering
                \textbf{OUTUBRO}\\ \vspace{0.15cm}
                \begin{tblr}{Q[c,m] Q[c,m] Q[c,m] Q[c,m] Q[c,m]}
                    \hline
                    \textbf{S} & \textbf{T} & \textbf{Q} & \textbf{Q} & \textbf{S} \\
                    \hline
                    &  & 01 & 02 & \aula03\\
                    06 & \aula07 & 08 & 09 & \aula10\\
                    13 & \aula14 & \feriado15 & 16 & \aula17\\
                    20 & \aula21 & 22 & 23 & \aula\esta{24}\\
                    27 & \aula28 & 29 & 30 & \prova31\\
                    \hline
                \end{tblr}
            \end{table}
        \end{column}
        
        \begin{column}{0.7\textwidth}
            \begin{itemize}
                \justifying
                \item Uso de \texttt{Task} e \texttt{Service} no JavaFX.
            \end{itemize}
        \end{column}
    \end{columns}
\end{frame}

\begin{frame}{Calendário}
    \centering
    \begin{tblr}{c c c}
        \aula AULA & \feriado FERIADO & \prova AVALIAÇÃO
    \end{tblr}
    
    \begin{columns}
        \begin{column}{0.3\textwidth}
            \begin{table}
                \centering
                \textbf{OUTUBRO}\\ \vspace{0.15cm}
                \begin{tblr}{Q[c,m] Q[c,m] Q[c,m] Q[c,m] Q[c,m]}
                    \hline
                    \textbf{S} & \textbf{T} & \textbf{Q} & \textbf{Q} & \textbf{S} \\
                    \hline
                    &  & 01 & 02 & \aula03\\
                    06 & \aula07 & 08 & 09 & \aula10\\
                    13 & \aula14 & \feriado15 & 16 & \aula17\\
                    20 & \aula21 & 22 & 23 & \aula24\\
                    27 & \aula\esta{28} & 29 & 30 & \prova31\\
                    \hline
                \end{tblr}
            \end{table}
        \end{column}
        
        \begin{column}{0.7\textwidth}
            \begin{itemize}
                \justifying
                \item Prática.
            \end{itemize}
        \end{column}
    \end{columns}
\end{frame}

\begin{frame}{Calendário}
    \centering
    \begin{tblr}{c c c}
        \aula AULA & \feriado FERIADO & \prova AVALIAÇÃO
    \end{tblr}
    
    \begin{columns}
        \begin{column}{0.3\textwidth}
            \begin{table}
                \centering
                \textbf{OUTUBRO}\\ \vspace{0.15cm}
                \begin{tblr}{Q[c,m] Q[c,m] Q[c,m] Q[c,m] Q[c,m]}
                    \hline
                    \textbf{S} & \textbf{T} & \textbf{Q} & \textbf{Q} & \textbf{S} \\
                    \hline
                    &  & 01 & 02 & \aula03\\
                    06 & \aula07 & 08 & 09 & \aula10\\
                    13 & \aula14 & \feriado15 & 16 & \aula17\\
                    20 & \aula21 & 22 & 23 & \aula24\\
                    27 & \aula28 & 29 & 30 & \prova\esta{31}\\
                    \hline
                \end{tblr}
            \end{table}
        \end{column}
        
        \begin{column}{0.7\textwidth}
            \Large\centering Segunda Avaliação
        \end{column}
    \end{columns}
\end{frame}

\begin{frame}{Calendário}
    \centering
    \begin{tblr}{c c c}
        \aula AULA & \feriado FERIADO & \prova AVALIAÇÃO
    \end{tblr}
    
    \begin{columns}
        \begin{column}{0.3\textwidth}
            \begin{table}
                \centering
                \textbf{NOVEMBRO}\\ \vspace{0.15cm}
                \begin{tblr}{Q[c,m] Q[c,m] Q[c,m] Q[c,m] Q[c,m]}
                    \hline
                    \textbf{S} & \textbf{T} & \textbf{Q} & \textbf{Q} & \textbf{S} \\
                    \hline
                    03 & \aula\esta{04} & 05 & 06 & \aula07\\
                    10 & \aula11 & 12 & 13 & \aula14\\
                    17 & \aula18 & 19 & \feriado20 & \aula21\\
                    24 & \aula25 & 26 & 27 & \aula28\\
                    \hline
                \end{tblr}
            \end{table}
        \end{column}
        
        \begin{column}{0.7\textwidth}
            \begin{itemize}
                \justifying
                \item Noções básicas de rede.
            \end{itemize}
        \end{column}
    \end{columns}
\end{frame}

\begin{frame}{Calendário}
    \centering
    \begin{tblr}{c c c}
        \aula AULA & \feriado FERIADO & \prova AVALIAÇÃO
    \end{tblr}
    
    \begin{columns}
        \begin{column}{0.3\textwidth}
            \begin{table}
                \centering
                \textbf{NOVEMBRO}\\ \vspace{0.15cm}
                \begin{tblr}{Q[c,m] Q[c,m] Q[c,m] Q[c,m] Q[c,m]}
                    \hline
                    \textbf{S} & \textbf{T} & \textbf{Q} & \textbf{Q} & \textbf{S} \\
                    \hline
                    03 & \aula04 & 05 & 06 & \aula\esta{07}\\
                    10 & \aula11 & 12 & 13 & \aula14\\
                    17 & \aula18 & 19 & \feriado20 & \aula21\\
                    24 & \aula25 & 26 & 27 & \aula28\\
                    \hline
                \end{tblr}
            \end{table}
        \end{column}
        
        \begin{column}{0.7\textwidth}
            \begin{itemize}
                \justifying
                \item JDBC.
            \end{itemize}
        \end{column}
    \end{columns}
\end{frame}

\begin{frame}{Calendário}
    \centering
    \begin{tblr}{c c c}
        \aula AULA & \feriado FERIADO & \prova AVALIAÇÃO
    \end{tblr}
    
    \begin{columns}
        \begin{column}{0.3\textwidth}
            \begin{table}
                \centering
                \textbf{NOVEMBRO}\\ \vspace{0.15cm}
                \begin{tblr}{Q[c,m] Q[c,m] Q[c,m] Q[c,m] Q[c,m]}
                    \hline
                    \textbf{S} & \textbf{T} & \textbf{Q} & \textbf{Q} & \textbf{S} \\
                    \hline
                    03 & \aula04 & 05 & 06 & \aula07\\
                    10 & \aula\esta{11} & 12 & 13 & \aula14\\
                    17 & \aula18 & 19 & \feriado20 & \aula21\\
                    24 & \aula25 & 26 & 27 & \aula28\\
                    \hline
                \end{tblr}
            \end{table}
        \end{column}
        
        \begin{column}{0.7\textwidth}
            \begin{itemize}
                \justifying
                \item Cliente TCP simples em JavaFX.
            \end{itemize}
        \end{column}
    \end{columns}
\end{frame}

\begin{frame}{Calendário}
    \centering
    \begin{tblr}{c c c}
        \aula AULA & \feriado FERIADO & \prova AVALIAÇÃO
    \end{tblr}
    
    \begin{columns}
        \begin{column}{0.3\textwidth}
            \begin{table}
                \centering
                \textbf{NOVEMBRO}\\ \vspace{0.15cm}
                \begin{tblr}{Q[c,m] Q[c,m] Q[c,m] Q[c,m] Q[c,m]}
                    \hline
                    \textbf{S} & \textbf{T} & \textbf{Q} & \textbf{Q} & \textbf{S} \\
                    \hline
                    03 & \aula04 & 05 & 06 & \aula07\\
                    10 & \aula11 & 12 & 13 & \aula\esta{14}\\
                    17 & \aula18 & 19 & \feriado20 & \aula21\\
                    24 & \aula25 & 26 & 27 & \aula28\\
                    \hline
                \end{tblr}
            \end{table}
        \end{column}
        
        \begin{column}{0.7\textwidth}
            \begin{itemize}
                \justifying
                \item Servidor TCP simples.
            \end{itemize}
        \end{column}
    \end{columns}
\end{frame}

\begin{frame}{Calendário}
    \centering
    \begin{tblr}{c c c}
        \aula AULA & \feriado FERIADO & \prova AVALIAÇÃO
    \end{tblr}
    
    \begin{columns}
        \begin{column}{0.3\textwidth}
            \begin{table}
                \centering
                \textbf{NOVEMBRO}\\ \vspace{0.15cm}
                \begin{tblr}{Q[c,m] Q[c,m] Q[c,m] Q[c,m] Q[c,m]}
                    \hline
                    \textbf{S} & \textbf{T} & \textbf{Q} & \textbf{Q} & \textbf{S} \\
                    \hline
                    03 & \aula04 & 05 & 06 & \aula07\\
                    10 & \aula11 & 12 & 13 & \aula14\\
                    17 & \aula\esta{18} & 19 & \feriado20 & \aula21\\
                    24 & \aula25 & 26 & 27 & \aula28\\
                    \hline
                \end{tblr}
            \end{table}
        \end{column}
        
        \begin{column}{0.7\textwidth}
            \begin{itemize}
                \justifying
                \item Conexão SQLite com JDBC em JavaFX.
            \end{itemize}
        \end{column}
    \end{columns}
\end{frame}

\begin{frame}{Calendário}
    \centering
    \begin{tblr}{c c c}
        \aula AULA & \feriado FERIADO & \prova AVALIAÇÃO
    \end{tblr}
    
    \begin{columns}
        \begin{column}{0.3\textwidth}
            \begin{table}
                \centering
                \textbf{NOVEMBRO}\\ \vspace{0.15cm}
                \begin{tblr}{Q[c,m] Q[c,m] Q[c,m] Q[c,m] Q[c,m]}
                    \hline
                    \textbf{S} & \textbf{T} & \textbf{Q} & \textbf{Q} & \textbf{S} \\
                    \hline
                    03 & \aula04 & 05 & 06 & \aula07\\
                    10 & \aula11 & 12 & 13 & \aula14\\
                    17 & \aula18 & 19 & \feriado20 & \aula\esta{21}\\
                    24 & \aula25 & 26 & 27 & \aula28\\
                    \hline
                \end{tblr}
            \end{table}
        \end{column}
        
        \begin{column}{0.7\textwidth}
            \begin{itemize}
                \justifying
                \item Inserção de registros no banco via GUI.
            \end{itemize}
        \end{column}
    \end{columns}
\end{frame}

\begin{frame}{Calendário}
    \centering
    \begin{tblr}{c c c}
        \aula AULA & \feriado FERIADO & \prova AVALIAÇÃO
    \end{tblr}
    
    \begin{columns}
        \begin{column}{0.3\textwidth}
            \begin{table}
                \centering
                \textbf{NOVEMBRO}\\ \vspace{0.15cm}
                \begin{tblr}{Q[c,m] Q[c,m] Q[c,m] Q[c,m] Q[c,m]}
                    \hline
                    \textbf{S} & \textbf{T} & \textbf{Q} & \textbf{Q} & \textbf{S} \\
                    \hline
                    03 & \aula04 & 05 & 06 & \aula07\\
                    10 & \aula11 & 12 & 13 & \aula14\\
                    17 & \aula18 & 19 & \feriado20 & \aula21\\
                    24 & \aula\esta{25} & 26 & 27 & \aula28\\
                    \hline
                \end{tblr}
            \end{table}
        \end{column}
        
        \begin{column}{0.7\textwidth}
            \begin{itemize}
                \justifying
                \item Atualização e exclusão de registros via GUI.
            \end{itemize}
        \end{column}
    \end{columns}
\end{frame}

\begin{frame}{Calendário}
    \centering
    \begin{tblr}{c c c}
        \aula AULA & \feriado FERIADO & \prova AVALIAÇÃO
    \end{tblr}
    
    \begin{columns}
        \begin{column}{0.3\textwidth}
            \begin{table}
                \centering
                \textbf{NOVEMBRO}\\ \vspace{0.15cm}
                \begin{tblr}{Q[c,m] Q[c,m] Q[c,m] Q[c,m] Q[c,m]}
                    \hline
                    \textbf{S} & \textbf{T} & \textbf{Q} & \textbf{Q} & \textbf{S} \\
                    \hline
                    03 & \aula04 & 05 & 06 & \aula07\\
                    10 & \aula11 & 12 & 13 & \aula14\\
                    17 & \aula18 & 19 & \feriado20 & \aula21\\
                    24 & \aula25 & 26 & 27 & \aula\esta{28}\\
                    \hline
                \end{tblr}
            \end{table}
        \end{column}
        
        \begin{column}{0.7\textwidth}
            \begin{itemize}
                \justifying
                \item Integração GUI + Banco + Rede.
            \end{itemize}
        \end{column}
    \end{columns}
\end{frame}

\begin{frame}{Calendário}
    \centering
    \begin{tblr}{c c c}
        \aula AULA & \feriado FERIADO & \prova AVALIAÇÃO
    \end{tblr}
    
    \begin{columns}
        \begin{column}{0.3\textwidth}
            \begin{table}
                \centering
                \textbf{DEZEMBRO}\\ \vspace{0.15cm}
                \begin{tblr}{Q[c,m] Q[c,m] Q[c,m] Q[c,m] Q[c,m]}
                    \hline
                    \textbf{S} & \textbf{T} & \textbf{Q} & \textbf{Q} & \textbf{S} \\
                    \hline
                    01 & \aula\esta{02} & 03 & 04 & \prova05\\
                    08 & \prova09 & 10 & 11 & \feriado12\\
                    15 & 16 & 17 & 18 & 19\\
                    22 & 23 & 24 & 25 & 26\\
                    29 & 30 & 31 &    &   \\
                    \hline
                \end{tblr}
            \end{table}
        \end{column}
        
        \begin{column}{0.7\textwidth}
            \begin{itemize}
                \item Prática.
            \end{itemize}
        \end{column}
    \end{columns}
\end{frame}

\begin{frame}{Calendário}
    \centering
    \begin{tblr}{c c c}
        \aula AULA & \feriado FERIADO & \prova AVALIAÇÃO
    \end{tblr}
    
    \begin{columns}
        \begin{column}{0.3\textwidth}
            \begin{table}
                \centering
                \textbf{DEZEMBRO}\\ \vspace{0.15cm}
                \begin{tblr}{Q[c,m] Q[c,m] Q[c,m] Q[c,m] Q[c,m]}
                    \hline
                    \textbf{S} & \textbf{T} & \textbf{Q} & \textbf{Q} & \textbf{S} \\
                    \hline
                    01 & \aula02 & 03 & 04 & \prova\esta{05}\\
                    08 & \prova09 & 10 & 11 & \feriado12\\
                    15 & 16 & 17 & 18 & 19\\
                    22 & 23 & 24 & 25 & 26\\
                    29 & 30 & 31 &    &   \\
                    \hline
                \end{tblr}
            \end{table}
        \end{column}
        
        \begin{column}{0.7\textwidth}
            \Large\centering Terceira Avaliação
        \end{column}
    \end{columns}
\end{frame}

\begin{frame}{Calendário}
    \centering
    \begin{tblr}{c c c}
        \aula AULA & \feriado FERIADO & \prova AVALIAÇÃO
    \end{tblr}
    
    \begin{columns}
        \begin{column}{0.3\textwidth}
            \begin{table}
                \centering
                \textbf{DEZEMBRO}\\ \vspace{0.15cm}
                \begin{tblr}{Q[c,m] Q[c,m] Q[c,m] Q[c,m] Q[c,m]}
                    \hline
                    \textbf{S} & \textbf{T} & \textbf{Q} & \textbf{Q} & \textbf{S} \\
                    \hline
                    01 & \aula02 & 03 & 04 & \prova05\\
                    08 & \prova\esta{09} & 10 & 11 & \feriado12\\
                    15 & 16 & 17 & 18 & 19\\
                    22 & 23 & 24 & 25 & 26\\
                    29 & 30 & 31 &    &   \\
                    \hline
                \end{tblr}
            \end{table}
        \end{column}
        
        \begin{column}{0.7\textwidth}
            \Large\centering Avaliação Final
        \end{column}
    \end{columns}
\end{frame}

%===============================================================================
% FIM ==========================================================================
%===============================================================================

\begin{frame}
    \centering
    \Large
    FIM
\end{frame}
    
\end{document}
